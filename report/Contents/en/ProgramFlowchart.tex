%%%%%%%%%%%%%%%%%%%%%%%%
%
% $Author: Sadegh Naderi $
% $Datum: 2024-01-16  $
% $Shodrt Description: A flowchart of the program is provided and explained $
%$Pfad: ML23-01-Keyword-Spotting-with-an-Arduino-Nano-33-BLE-Sense\report\Contents\en\ProgramFlowchart.tex $
% $Version: 2.0 $
% $Review by: Sadegh Naderi $
% $Review date: 2024-02-12 $
%
%%%%%%%%%%%%%%%%%%%%%%%%


\chapter{Program Flowchart}
\label{chapter:ProgramFlowchart}

In this Chapter, a breakdown of each step involved in the keyword spotting program is provided. The program flowchart is shown in Figure \ref{fig:progFlowchart}. For a more detailed explanation of the program see Section \ref{section:MLPipeline} and for understanding the program tests visit chapter \ref{chapter:SoftwareTests}.

\begin{itemize}
	\item \textbf{Start:} The program initiates the keyword spotting process.
	
	\item \textbf{Dataset Existence Check:} The program checks if the dataset is already available. If present, it proceeds to the "Load Dataset" step; otherwise, it initiates the process of downloading and extracting the dataset.
	
	\item \textbf{Load Dataset:} The dataset is loaded from the specified path. A test is conducted to verify the successful loading of the dataset without errors.
	
	\item \textbf{Audio Preprocessing:} The loaded audio dataset undergoes preprocessing, specifically squeezing the audio to remove unnecessary dimensions. A test validates the correct execution of audio preprocessing steps.
	
	\item \textbf{Create Spectrogram Dataset:} Spectrograms are generated from the preprocessed audio data, transforming audio signals into frequency representations. The process is tested to confirm spectrogram creation.
	
	\item \textbf{Build Model:} The neural network model is constructed with specific layers, configurations, and input shapes. A test ensures that the model is built correctly and ready for compilation.
	
	\item \textbf{Compile Model:} The model is compiled by specifying the optimizer, loss function, and evaluation metrics, preparing it for training.
	
	\item \textbf{Train Model:} The model is trained on the training dataset, involving shuffling, caching, and prefetching for efficiency. The training process is monitored, and the model's performance is evaluated on the validation dataset.
	
	\item \textbf{Create Test Dataset:} A separate dataset is created for testing the model's predictions, simulating real-world scenarios.
	
	\item \textbf{Evaluate Test Predictions:} The model is evaluated on the test dataset to assess its predictive accuracy and generalization to new, unseen data.
	
	\item \textbf{Save Model:} The trained model is saved to disk for future use or deployment, preserving its architecture and learned parameters.
	
	\item \textbf{Convert to TFLite and Save:} The saved model is converted into TensorFlow Lite format, a lightweight format suitable for deployment on resource-constrained devices. The converted model is then saved.
	
	\item \textbf{Test Model Export:} The exporting functions are tested to ensure that the program is capable of exporting the model.
	
	\item \textbf{Catch and Handle Errors:} The program includes an error-handling mechanism to capture and handle any unexpected errors during the entire process, enhancing resilience to potential issues.
	
	\item \textbf{Log the Errors:} In case of an error, information about the error is logged. This step aids in debugging and improving the program by providing insights into the nature of encountered issues.
	
	\item \textbf{End:} The program concludes, marking the successful completion of the keyword spotting pipeline.
\end{itemize}


\begin{figure}[h!]
	\centering
	%%%%%%%%%%%%%%%%%%%%%%%%
%
% $Author: Sadegh Naderi $
% $Datum: 11.02.2024  $
% $Pfad: ML23-01-Keyword-Spotting-with-an-Arduino-Nano-33-BLE-Sense\report\Images\Flowchart\programmingFlowchart.tex $
% $Version: 1.0 $
% $Reviewed by: Sadegh Naderi $
% $Review Date: 11.02.2024 $
%
% !TeX encoding = utf8
%
%%%%%%%%%%%%%%%%%%%%%%%%


\begin{tikzpicture}[>=stealth, node distance=1cm, every node/.style={rectangle, draw, align=center, font=\footnotesize}]
	
	% Nodes
	\node (start) [ellipse, fill=orange!30] {Start};
	\node (dataset) [diamond, below of=start, yshift=-0.9cm, fill=green!30] {Dataset \\ exists?};
	\node (load) [below of=dataset, yshift=-2cm, fill=blue!30] {Load Dataset};
	\node (download) [right=2cm of dataset, fill=blue!30] {Download Dataset};
	\node (extract) [below of=download, yshift=-2cm, fill=blue!30] {Extract Dataset};
	\node (testload) [below of=load, fill=blue!30] {Test Data Loading};
	\node (preprocess) [below of=testload, fill=blue!30] {Preprocess Audio Dataset};
	\node (testpreprocess) [below of=preprocess, fill=blue!30] {Test Audio Preprocessing};
	\node (create) [below of=testpreprocess, fill=blue!30] {Create Spectrogram Dataset};
	\node (testspectrogram) [below of=create, fill=blue!30] {Test Spectrogram Creation};
	\node (build) [below of=testspectrogram, fill=blue!30] {Build Model};
	\node (testbuild) [below of=build, fill=blue!30] {Test Model Building};
	\node (compile) [below of=testbuild, fill=blue!30] {Compile Model};
	\node (train) [below of=compile, fill=blue!30] {Train Model};
	\node (evaluate) [below of=train, fill=blue!30] {Evaluate Model};
	\node (createtest) [below of=evaluate, fill=blue!30] {Create Test Dataset};
	\node (evaluatetest) [below of=createtest, fill=blue!30] {Evaluate Test Prediction};
	\node (save) [below of=evaluatetest, fill=blue!30] {Save Model};
	\node (convert) [below of=save, fill=blue!30] {Convert to TFLite and Save};
	\node (testexport) [below of=convert, fill=blue!30] {Test Model Export};
	\node (errors) [below of=testexport, fill=blue!30] {Catch and Handle Errors};
	\node (errorslog) [below of=errors, fill=blue!30] {Log the Errors};
	\node (end) [ellipse, below of=errorslog, fill=orange!20] {End};
	
	% Arrows
	\draw[->] (start) -- (dataset);
	\draw[->] (dataset) -- node[left, draw=none] {yes} (load);
	\draw[->] (dataset) -- node[above, draw=none] {no} (download);
	\draw[->] (download) -- (extract);
	\draw[->] (extract) -- (load);
	\draw[->] (load) -- (testload);
	\draw[->] (testload) -- (preprocess);
	\draw[->] (preprocess) -- (testpreprocess);
	\draw[->] (testpreprocess) -- (create);
	\draw[->] (create) -- (testspectrogram);
	\draw[->] (testspectrogram) -- (build);
	\draw[->] (build) -- (testbuild);
	\draw[->] (testbuild) -- (compile);
	\draw[->] (compile) -- (train);
	\draw[->] (train) -- (evaluate);
	\draw[->] (evaluate) -- (createtest);
	\draw[->] (createtest) -- (evaluatetest);
	\draw[->] (evaluatetest) -- (save);
	\draw[->] (save) -- (convert);
	\draw[->] (convert) -- (testexport);
	\draw[->] (testexport) -- (errors);
	\draw[->] (errors) -- (errorslog);
	\draw[->] (errorslog) -- (end);
	
\end{tikzpicture}

	\caption{The program flowchart} \label{fig:progFlowchart}
\end{figure}

